\documentclass[a4paper,12pt]{article}

%%%Преамбула

\usepackage{float}

\usepackage{indentfirst} %Красная строка первого абзаца после заголовка

%структура article следующая:
%part-section-subsection-subsubsection-paragraph-subparagraph

%%%Работа с русским языком
\usepackage{cmap}					% поиск в PDF
%\usepackage{mathtext}			%русские буквы в формулах. КИРИЛЛИЦА НЕ ПРЕДУСМОТРЕНА В ЛАТЕХЕ - лучше отключить
\usepackage[T2A]{fontenc}			% кодировка
\usepackage[utf8]{inputenc}			% кодировка исходного текста
\usepackage[english,russian]{babel}	% локализация и переносы

%%%Шрифты
\usepackage{euscript}		%шрифт Евклид
\usepackage{mathrsfs}		%красивый шрифт

%%%Доп работа с математикой
\usepackage{amsmath,amsfonts,amssymb,amsthm,mathtools} %AMS - корень матана
\usepackage{cases}
\usepackage{icomma} % для отображения запятой как разделителя целой и дробной частей числа

%%%Номера формул
\mathtoolsset{showonlyrefs=true}%Показывать номера только у тех формул, на которые есть \eqref{} в тексте.

%%%Свои команды
\DeclareMathOperator{\sgn}{\mathop{sgn}}
%СИНТАКСИС: \деклейроператор{названиеКоманды}{чтоНужноСделатьКоманде}

%%%Перенос знаков в формулах (по Львовскому) стр 54 мануала
%Т.е. если формула имеет разрыв и переходит на новую строку, то по нормам русского языка арифметический знак должен дублироваться (Латех подчиняется американским нормам, где такого правила нет).
\newcommand*{\hm}[1]{#1\nobreak\discretionary{}%
{\hbox{$\mathsurround=0pt #1$}}{}}

%%%Графика
\usepackage{graphicx} %для \includegraphics{•}
%\usepackage[lofdepth,lotdepth]{subfig}
\graphicspath{{images/}} %теперь пикчи ТеХ будет брать из соответствующей папки в своей домашней папке
\setlength\fboxsep{3pt} %отступ рамки \fbox{} от рисунка
\setlength\fboxrule{1pt} %толщина линий рамки \fbox{}
\usepackage{wrapfig} %обтекание рисунков и таблиц текстом
\usepackage{subcaption}

%%%Таблицы
\usepackage{array,tabularx,tabulary,booktabs} %Доп работа с таблицами
\usepackage{longtable} %длинные таблицы
\usepackage{multirow} %слияние строк в таблице

%%% Программирование
\usepackage{etoolbox} % логические операторы

%%% Страница
%\usepackage{extsizes} % Возможность сделать 14-й шрифт
\usepackage{geometry} % Простой способ задавать поля
	\geometry{top=25mm}
	\geometry{bottom=35mm}
	\geometry{left=35mm}
	\geometry{right=20mm}

%\usepackage{fancyhdr} % Колонтитулы
% 	\pagestyle{fancy}
% 	\renewcommand{\headrulewidth}{0mm}  % Толщина линейки, отчеркивающей верхний колонтитул
 	% \lfoot{Нижний левый}
 	% \rfoot{Нижний правый}
 	% \rhead{Верхний правый}
 	% \chead{Верхний в центре}
 	% \lhead{Верхний левый}
 	% \cfoot{Нижний в центре} % По умолчанию здесь номер страницы

\usepackage{setspace} % Интерлиньяж
\onehalfspacing % Интерлиньяж 1.5
%\doublespacing % Интерлиньяж 2
%\singlespacing % Интерлиньяж 1

\usepackage{lastpage} % Узнать, сколько всего страниц в документе.

\usepackage{soulutf8} % Модификаторы начертания

\usepackage{hyperref}
\usepackage[usenames,dvipsnames,svgnames,table,rgb]{xcolor}
\hypersetup{				% Гиперссылки
    unicode=true,           % русские буквы в раздела PDF
    pdftitle={Заголовок},   % Заголовок
    pdfauthor={Автор},      % Автор
    pdfsubject={Тема},      % Тема
    pdfcreator={Создатель}, % Создатель
    pdfproducer={Производитель}, % Производитель
    pdfkeywords={keyword1} {key2} {key3}, % Ключевые слова
    colorlinks=true,       	% false: ссылки в рамках; true: цветные ссылки
    linkcolor=red,          % внутренние ссылки
    citecolor=green,        % на библиографию
    filecolor=magenta,      % на файлы
    urlcolor=cyan           % на URL
}

%\renewcommand{\familydefault}{\sfdefault} % Начертание шрифта

\usepackage{multicol} % Несколько колонок

\usepackage{enumitem} % Жирные маркеры в begin{enumerate}

%%%Библиография

%\usepackage{cite} % Работа с библиографией (НЕСОВМЕСТИМ С БИБЛАТЕХ)
%\usepackage[superscript]{cite} % Ссылки в верхних индексах
%\usepackage[nocompress]{cite} % 

\usepackage[backend=bibtex, bibencoding=utf8, sorting=ynt, maxcitenames=2, style=numeric]{biblatex}

\usepackage{csquotes} % Еще инструменты для ссылок

\usepackage{multicol} % Несколько колонок